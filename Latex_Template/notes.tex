\documentclass[12pt]{article}
\usepackage{amsmath,amsthm,amsfonts,amssymb,amscd}
\usepackage{multirow,booktabs}
\usepackage[table]{xcolor}
\usepackage{fullpage}
\usepackage{lastpage}
\usepackage{enumitem}
\usepackage{fancyhdr}
\usepackage{mathrsfs}
\usepackage{wrapfig}
\usepackage{setspace}
\usepackage{calc}
\usepackage{multicol}
\usepackage{cancel}
\usepackage[retainorgcmds]{IEEEtrantools}
\usepackage[margin=3cm]{geometry}
\usepackage{amsmath}
\newlength{\tabcont}
\setlength{\parindent}{0.0in}
\setlength{\parskip}{0.05in}
\usepackage{empheq}
\usepackage{framed}
\usepackage[most]{tcolorbox}
\usepackage{xcolor}
\colorlet{shadecolor}{orange!15}
\parindent 0in
\parskip 12pt
\geometry{margin=1in, headsep=0.25in}
\theoremstyle{definition}
\newtheorem{defn}{Definition}
\newtheorem{reg}{Rule}
\newtheorem{exer}{Exercise}
\newtheorem{note}{Note}
\thispagestyle{empty}

\begin{document}
\setcounter{section}{0}

\begingroup
    \centering
    \LARGE CPSC 370: Introduction To Computer Cryptology\\[0.5em]
    \large \today\\[0.5em]
    \large Aiden Rivera\par
\endgroup

%\section{Section}
%\subsection{Subsection}
%
%NOTE
%\begin{note}
%\textbf{\textit{S} S$_0$}
%\end{note}
%
%EQUATION
%\begin{equation}
%F = m\ddot{r_0}
%\end{equation}
%
%SHADED
%\begin{shaded}
%\textbf{The Tidal Force} \newline
%\begin{equation}
%F_{tide} = -GM_mm(\frac{\hat{d}}{d^2}-\frac{\hat{d_0}}{d_0^2})
%\end{equation}
%Where:
%\begin{equation*}
%\begin{split}
%G = \text{Gravitational Constant} \\
%d = \text{Object's Position Relative to Moon} \\
%d_0 = \text{Earth's Center Relative to the moon}\\
%M_m = \text{Mass of the moon}
%\end{split}
%\end{equation*}
%\end{shaded}
%
%DEFINITION
%\begin{defn}
%\textbf{Euler's Theorem} - The most general motion of any body relative to a fixed point \textit{O} is a rotation about some axis through \textit{O} To specify this rotation about a given point O, we only have to give the direction of the axis and the rate of rotation, or angular velocity $\omega$. Because this has a magnitude and direction, it is an obvious choice to write this rotation vector as $\omega$, the angular velocity vector. That is:
%\begin{equation}
%\omega = \omega\textbf{u}
%\end{equation}
%Where \textbf{u} is the unit vector
%\end{defn}
\end{document}
