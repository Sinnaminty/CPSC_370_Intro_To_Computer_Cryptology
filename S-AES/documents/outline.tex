\documentclass{article}

\usepackage{booktabs} % for toprule, midrule, bottomrule in tables
\usepackage{authblk} % for multiple authors and affiliations
\usepackage{graphicx} % to include graphics with \includegraphics

\author[1]{Aiden Rivera}

\title{Simplified AES Under Attack: A Peer into Differential Cryptanalysis}
\begin{document}
\maketitle
\tableofcontents
\newpage

\section{Introduction}
Simplified AES (S-AES) is a lightweight version of the Advanced Encryption Standard (AES) designed for educational purposes. Despite its simplicity, S-AES retains the fundamental structure of AES, including the substitution-permutation network, and provides a practical framework for studying cryptographic concepts. My research explores the application of differential cryptanalysis, a prominent method for analyzing block ciphers, to the S-AES cipher.

To contextualize the attack, I implemented a two-round version of S-AES encryption and decryption in software and studied the statistical properties of differential propagation through its structure. Differential cryptanalysis leverages high-probability differences in plaintexts and their resulting ciphertexts to deduce parts of the secret key. While this technique is powerful against many lightweight ciphers, the limited 16-bit keyspace of S-AES makes brute-force attacks computationally trivial in practice. As a result, differential cryptanalysis is more of an analytical tool for understanding this particular cipher's weaknesses rather than a practical method for key recovery.

\section{Research Objectives}
The primary objective of this research is to investigate the feasibility and effectiveness of applying differential cryptanalysis to S-AES. Specifically, the study aims to:

\subsection{Analyze Differential Properties}
Examine the propagation of input differences through the two-round S-AES encryption process to identify statistical patterns that can reveal key information.

\subsection{Evaluate the Practicality of Differential Cryptanalysis}    
Assess the computational requirements of differential cryptanalysis when applied to S-AES, particularly in comparison to a brute-force approach, given the cipher’s small 16-bit keyspace.

\subsection{Understand the Latent Cryptographic Weaknesses}
Explore the structural vulnerabilities in S-AES that make it susceptible to differential cryptanalysis, providing insights into how differential attacks exploit the cipher’s design.

\newpage

\section{Key Elements}
The main takeaways of my research lie in the following objectives, methodologies and findings.

\subsection{Objectives}
\begin{itemize}
    \item \textbf{Investigate Differential Cryptanalysis:} Analyze how differential cryptanalysis applies to the S-AES cipher, focusing on the propagation of plaintext differences through its structure.

    \item \textbf{Compare Cryptanalysis and Brute Force:} Evaluate the practicality of differential cryptanalysis versus brute force for breaking S-AES's small 16-bit keyspace.
\end{itemize}

\subsection{Methodology}
\begin{itemize}
    \item \textbf{Implementation of S-AES:} Developed a two-round implementation of S-AES encryption and decryption to understand its internal mechanics and to simulate key recovery experiments.

    \item \textbf{Study of Differential Properties:} Researched how input differences propagate through the S-boxes, shift rows, and MixColumns transformations of S-AES, building a theoretical basis for differential attacks.

    \item \textbf{Statistical Analysis of Differentials:} Examined differential distribution tables to identify high-probability input-output pairs that could be used to deduce key information.

    \item \textbf{Theoretical Key Recovery:} Outlined the process of using differential pairs to reduce the key search space, though this was not implemented, as brute-force key search was computationally cheaper.
\end{itemize}

\subsection{Findings}
\begin{itemize}
    \item \textbf{Feasibility of Differential Cryptanalysis:} Differential cryptanalysis is theoretically applicable to S-AES, but not practical. The small 16-bit keyspace of S-AES renders brute-force attacks significantly faster and more practical than implementing a full differential attack.
\end{itemize}
\end{document}